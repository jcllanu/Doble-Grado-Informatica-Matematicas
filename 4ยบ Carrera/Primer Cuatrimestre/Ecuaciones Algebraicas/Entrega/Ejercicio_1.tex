% Aquí comienza la resolución del ejercicio 1

\textbf{1.} \textit{Sea $u := \sqrt{2+\sqrt{2+\sqrt{2}}}$. Demostrar que $\mathbb{Q}(u)|\mathbb{Q}$ es una extensión de Galois y calcular el grupo de Galois $G(\mathbb{Q}(u):\mathbb{Q})$.}

Un polinomio anulador de $u$ es $P(\mathtt{t}) = \left(\left( \mathtt{t}^2-2\right)^2-2\right)^2-2 = \mathtt{t}^8-8\mathtt{t}^6+20\mathtt{t}^4-16\mathtt{t}^2+2$, que es mónico y además irreducible en $\mathbb{Z}[ \mathtt{t}] $ por el Criterio de Eisenstein con el primo 2, lo que es equivalente a su irreducibilidad en $\mathbb{Q}[\mathtt{t}]$ por el Lema de Gauss. Por tanto, se tiene que $P$ es el polinomio mínimo de $u$ sobre $\mathbb{Q}$.

Además, si $\delta_1, \delta_2, \delta_3 \in \{-1,1\}$, tenemos que $\delta_1\sqrt{2+\delta_2\sqrt{2+\delta_3\sqrt{2}}}$ es también raíz de $P$ para cualquier combinación de los $\delta_i$, ya que $(\delta_i)^2 = 1$. Por tanto, para ver que la extensión $\mathbb{Q}(u)|\mathbb{Q}$ es de Galois, debemos probar que las 8 raíces se pueden escribir en función de $u$. Nos centraremos únicamente en aquellas tales que $\delta_1 = 1$, ya que las otras se pueden obtener multiplicando la expresión resultante por $-1$.

Comenzamos hallando las expresiones de $\sqrt{2}$, $r := \sqrt{2+\sqrt{2}}$ y $s := \sqrt{2-\sqrt{2}}$, que nos serán útiles más adelante:
\begin{equation*}
\begin{split}
&r = u^2-2\\
&\sqrt{2} = r^2-2 = u^4-4u^2+2\\
& s = \frac{\sqrt{2}}{r} = \frac{u^4-4u^2+2}{u^2-2} \stackrel{(*)}{=} \frac{u^8-8u^6+21u^4-20u^2+4}{u^2-2} = u^6-6u^4+9u^2-2\\
\end{split}
\end{equation*}
En $(*)$ hemos hecho uso de que $u$ es raíz de $P$, es decir, $-2 = u^8-8u^6+20u^4-16u^2$. Para aplicar la igualdad hemos sumado y restado 2 en el numerador y sustituido el $-2$.

Con esto, podemos calcular el resto de raíces positivas de $P$, que denotaremos por:
$$
v := \sqrt{2-\sqrt{2+\sqrt{2}}} \hspace{30pt}
w := \sqrt{2+\sqrt{2-\sqrt{2}}} \hspace{30pt}
x := \sqrt{2-\sqrt{2-\sqrt{2}}}
$$
Entonces, la expresión de $v$ será:
$$ v = \frac{s}{u} = \frac{u^6-6u^4+9u^2-2}{u} $$
Podemos calcular $w$ teniendo en cuenta que $w=\sqrt{2+s}$, por lo que, como $u > 0$, se tiene:
$$ w = \sqrt{2+s} = \sqrt{2+\left(u^6-6u^4+9u^2-2\right)} = \sqrt{u^6-6u^4+9u^2} = \left| u^3-3u\right| = u \left| u^2-3\right|$$
Como $\sqrt{2}>1$, tenemos que $r=\sqrt{2+\sqrt{2}}>\sqrt{2}>1$, por lo que $u^2 = 2+r > 3$, y por tanto:
$$ w = u \left| u^2-3\right| = u \left(u^2-3\right) = u^3-3u $$
Finalmente, podemos obtener $x$ a partir de la expresión de $w$:
$$ x = \frac{r}{w} = \frac{u^2-2}{u^3-3u}
% \stackrel{(*)}{=} \frac{u^8-8u^6+20u^4-15u^2}{u^3-3u} = u^5-5u^3+5u
$$
Puesto que hemos podido escribir todas las raíces positivas en función de $u$, concluimos que $v,w,x \in \mathbb{Q}(u)$, por lo que $\mathbb{Q}(u)$ es un cuerpo de descomposición de $P$, y por tanto la extensión $\mathbb{Q}(u)|\mathbb{Q}$ es de Galois.

Pasamos ahora a calcular el grupo de Galois $G(\mathbb{Q}(u):\mathbb{Q})$. Sea $f$ el $\mathbb{Q}$-automorfismo de $\mathbb{Q}(u)$ tal que $f(u) = -w = -u^3+3u$. Veamos que el orden de $f$ es 8, por lo que $G(\mathbb{Q}(u):\mathbb{Q})$ será cíclico de orden 8, es decir, isomorfo a $\mathbb{Z}_8$. Para ello, calculamos:
\begin{equation*}
\begin{split}
& f(w) = f\left(u^3-3u\right) = f(u)^3-3f(u) = -w^3+3w = v\\
& f(v) = f\left(\frac{w^2-2}{u}\right) = \frac{f(w^2-2)}{f(u)} = \frac{f(w)^2-2}{-w} = \frac{v^2-2}{-w} = \frac{\sqrt{2+\sqrt{2}}}{\sqrt{2+\sqrt{2-\sqrt{2}}}} = x \\
& f(x) = f\left(\frac{u^2-2}{w}\right) = \frac{f(u)^2-2}{f(w)} = \frac{w^2-2}{v} = \frac{\sqrt{2-\sqrt{2}}}{\sqrt{2-\sqrt{2+\sqrt{2}}}} = u \\
\end{split}
\end{equation*}
donde la única comprobación que debemos hacer es que efectivamente $v=-w^3+3w$:
$$ v = -w^3+3w = w(3-w^2) = \left(\sqrt{2+\sqrt{2-\sqrt{2}}}\right)\left(1-\sqrt{2-\sqrt{2}}\right) > 0 $$
Elevando ambos lados al cuadrado (tanto $v$ como $-w^3+3w$ son positivos, por lo que mantenemos la equivalencia) obtenemos que:
\begin{equation*} 
\begin{split}
	v^2 = 2-\sqrt{2+\sqrt{2}} &= \left(2+\sqrt{2-\sqrt{2}}\right)\left(3-\sqrt{2}-2\sqrt{2-\sqrt{2}}\right) = \\
	                          &= 2- \sqrt{2-\sqrt{2}}-\sqrt{2}\sqrt{2-\sqrt{2}}
\end{split}
\end{equation*}
Restando $2$ en cada lado, multiplicando por $-1$ y simplificando:
$$ \sqrt{2+\sqrt{2}} = \sqrt{2-\sqrt{2}} + \sqrt{2}\sqrt{2-\sqrt{2}} = \sqrt{2-\sqrt{2}}\left(1+\sqrt{2}\right) $$
Finalmente, dividiendo entre $\sqrt{2-\sqrt{2}}$ llegamos a que:
\begin{equation*} 
\begin{split}
	\frac{\sqrt{2+\sqrt{2}}}{\sqrt{2-\sqrt{2}}} = \frac{\sqrt{2+\sqrt{2}}}{\sqrt{2-\sqrt{2}}}\frac{\sqrt{2+\sqrt{2}}}{\sqrt{2+\sqrt{2}}} = \frac{2+\sqrt{2}}{\sqrt{2}} = 1+\sqrt{2}
\end{split}
\end{equation*}
lo cual es cierto, y por tanto $-w^3+3w = v$. 

Con esto, y teniendo en cuenta que $f(-k)=-f(k)$ para $k = u,v,w,x$ se comprueba que $f$ tiene orden 8 puesto que hemos probado que:
\begin{equation*}
	u \xrightarrow{f} -w \xrightarrow{f} -v \xrightarrow{f} -x \xrightarrow{f} -u \xrightarrow{f} w \xrightarrow{f} v \xrightarrow{f} x \xrightarrow{f} u
\end{equation*}
