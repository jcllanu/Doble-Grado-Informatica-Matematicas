% Aquí comienza la resolución del ejercicio 3
\unaccentedoperators
\textbf{3.} (i) \textit{Sea $\zeta$ := $e^{2\pi \mathtt{i}/5}$, donde $\mathtt{i}$ := $\sqrt{-1}$. Demostrar que $\sqrt{5}\in \mathbb{Q}(\zeta)$.}
\\
(ii) \textit{Sea $E:=\mathbb{Q}(\sqrt{5})$. Calcular el grado de las extensiones $E\left(\sqrt[10]{5}\right)|E$ y $E\left(e^{2\pi \mathtt{i}/10}\right)|E$.}
\\
(iii) \textit{Sea $f(\mathtt{t}):=\mathtt{t}^{10}-5$ y $\mathbb{Q}_f \subset \mathbb{C}$ un cuerpo de descomposición de f sobre $\mathbb{Q}$. Calcular el grado de $\mathbb{Q}_f | \mathbb{Q}$.}
\\
(iv) \textit{¿Cuántas subextensiones de grado $5$ tiene $\mathbb{Q}_f | \mathbb{Q}$?}
\\
(v) \textit{Demostrar que $\sqrt{3} \notin \mathbb{Q}(\zeta)$.}
\\
(vi) \textit{Calcular el polinomio mínimo de $\zeta$ sobre $\mathbb{Q}(\sqrt{3})$.}

(i) Podemos expresar $\zeta$ como $\zeta$ = $\cos{\left(\frac{2\pi}{5}\right)} + \mathtt{i}\sin{\left(\frac{2\pi}{5}\right)}$. Sea $u:=\zeta + \zeta^{-1}$. Por la fórmula de Euler para $\zeta$ y  $\zeta^{-1}$ sabemos que $u = \zeta + \zeta^{-1}$ = $2\cos{\left(\frac{2\pi}{5}\right)} \in \mathbb{Q}\left(\zeta\right)$.

Como $\zeta^{5}=1$ y $\zeta\neq1$, $\zeta$ será raíz del polinomio ciclotómico $\Phi_{5}\left(\mathtt{t}\right) = \mathtt{t}^4+\mathtt{t}^3+\mathtt{t}^2+\mathtt{t}+1$. Sea $h\left(\mathtt{t}\right) := \mathtt{t}^{-2}\cdot\Phi_{5}(\mathtt{t})$, que se anula en $\zeta$ ya que $\zeta$ no es raíz de $\mathtt{t}^{2}$. Reescribiendo $h$ llegamos a que:

$h\left(\mathtt{t}\right) = \left(\mathtt{t}^2+\mathtt{t}^{-2}\right)+\left(\mathtt{t}+\mathtt{t}^{-1}\right)+1$ y, como $\left(\mathtt{t}+\mathtt{t}^{-1}\right)^2 = \mathtt{t}^2+\mathtt{t}^{-2}+2$, entonces:

$h\left(\mathtt{t}\right) = \left[\left(\mathtt{t}+\mathtt{t}^{-1}\right)^2-2\right]+\left(\mathtt{t}+\mathtt{t}^{-1}\right)+1 = \left(\mathtt{t}+\mathtt{t}^{-1}\right)^{2} + \left(\mathtt{t}+\mathtt{t}^{-1}\right) - 1$

Como $h\left(\zeta\right) = 0$, deducimos que $h\left(\zeta\right) = \left(\zeta + \zeta^{-1}\right)^{2}+\left(\zeta + \zeta^{-1}\right)-1 = 0 \Leftrightarrow u^{2}+u-1 = 0$

Resolviendo la ecuación de segundo grado, tenemos que $u = \frac{-1\pm \sqrt{5}}{2}$. Dado que $u$ ha de ser positivo, tomamos la solución positiva, obteniendo que $\sqrt{5} = 2u+1 \in\mathbb{Q}\left(\zeta\right)$.

(ii) En primer lugar, $E\left(\sqrt[10]{5}\right) = \mathbb{Q}\left(\sqrt[10]{5}\right)$ ya que $\sqrt{5}\in\mathbb{Q}\left(\sqrt[10]{5}\right)$, pues $\left(\sqrt[10]{5}\right)^5 = \sqrt{5}$. Es decir, $\mathbb{Q}\left(\sqrt{5}\right)|\mathbb{Q}$ es una subextensión de $\mathbb{Q}\left(\sqrt[10]{5}\right)|\mathbb{Q}$. Sabemos que $[\mathbb{Q}\left(\sqrt{5}\right):\mathbb{Q}] = 2$ ya que el polinomio mínimo de $\sqrt{5}$ sobre $\mathbb{Q}$ es $g_1\left(\mathtt{t}\right) := \mathtt{t}^2 - 5$ (es mónico e irreducible por el Criterio de Eisenstein con el primo 5, aplicable a $\mathbb{Q}$ por el Lema de Gauss) y también sabemos que $[\mathbb{Q}\left(\sqrt[10]{5}\right):\mathbb{Q}] = 10$ ya que el polinomio mínimo de $\sqrt[10]{5}$ sobre $\mathbb{Q}$ es $g_2\left(\mathtt{t}\right) := \mathtt{t}^{10} - 5$ que de nuevo es mónico e irreducible argumentando de forma análoga. Por todo lo anterior y la transitividad del grado:
$$[E(\sqrt[10]{5}):E] = [\mathbb{Q}(\sqrt[10]{5}):\mathbb{Q}(\sqrt{5})] = \frac{[\mathbb{Q}\left(\sqrt[10]{5}\right):\mathbb{Q}]}{[\mathbb{Q}\left(\sqrt{5}\right):\mathbb{Q}]} = \frac{10}{2} = 5$$
Argumentando de forma similar, como $e^{2\pi \mathtt{i}/5}=\left(e^{2\pi \mathtt{i}/10}\right)^{2}$ y $\sqrt{5}\in \mathbb{Q}\left(e^{2\pi \mathtt{i}/5}\right)$ según lo visto en (i), entonces $\sqrt{5}\in \mathbb{Q}\left(e^{2\pi \mathtt{i}/10}\right)$. Por tanto, $E\left(e^{2\pi \mathtt{i}/10}\right) = \mathbb{Q}\left(e^{2\pi \mathtt{i}/10}\right)$. Sabemos que el polinomio mínimo de $e^{2\pi \mathtt{i}/10}$ sobre $\mathbb{Q}$ es el polinomio ciclotómico $\Phi_{10}\left(\mathtt{t}\right) = \mathtt{t}^4-\mathtt{t}^3+\mathtt{t}^2-\mathtt{t}+1$, que tiene grado $4=\varphi(10)$, por lo que:
$$[E(e^{2\pi \mathtt{i}/10}):E] = [\mathbb{Q}(e^{2\pi \mathtt{i}/10}):\mathbb{Q}(\sqrt{5})] = \frac{[\mathbb{Q}(e^{2\pi \mathtt{i}/10}):\mathbb{Q}]}{[\mathbb{Q}(\sqrt{5}):\mathbb{Q}]} = \frac{4}{2} = 2$$

(iii) Sea $r:=\sqrt[10]{5}$ la única raíz real positiva de $f$. El resto de raíces serán de la forma $r\cdot\xi^{k}$ donde $0\leq k\leq 9$ y $\xi := e^{2\pi \mathtt{i}/10}$ por lo que el conjunto de raíces de $f$ será $\{r\cdot\xi^{k}: 0\leq k\leq 9\}$. Por tanto, un cuerpo de descomposición de $f$ sobre $\mathbb{Q}$ es $\mathbb{Q}_f = \mathbb{Q}\left(r,\xi\right)$. Ya sabemos por el apartado anterior que $[E\left(\xi\right):E] = 2$ y $[E\left(r\right):E] = 5$ y como 5 y 2 son coprimos, $[E\left(r,\xi\right):E]=5\cdot 2=10$. Por último, como $[\mathbb{Q}\left(\sqrt{5}\right):\mathbb{Q}] = 2$ y $\sqrt{5}\in \mathbb{Q}\left(\xi\right)$ concluimos que  $[\mathbb{Q}_f:\mathbb{Q}]=[\mathbb{Q}\left(r,\xi\right):\mathbb{Q}]=[E\left(r,\xi\right):\mathbb{Q}]=[E\left(r,\xi\right):E]\cdot[E:\mathbb{Q}]=10\cdot 2=20$.

(iv) Como la extensión  $\mathbb{Q}_f | \mathbb{Q}$ es de Galois, su grado (que es 20) coincide con el orden de $G := G(\mathbb{Q}_f : \mathbb{Q})$. Así que el número de subextensiones de grado $5$ coincidirá con el número de subgrupos de $G$ de orden $4 = \frac{20}{5}$. Puesto que $20 = 2^2 \cdot 5$, los subgrupos de $G$ de orden $4$ son sus 2-subgrupos de Sylow. Por el tercer Teorema de Sylow, si $n_2$ es el número de 2-subgrupos de Sylow de $G$, entonces $n_2 \equiv1\mod 2$ y $n_2 | 5$. Es decir, $n_2$ es o bien $1$, o bien $5$.

Veamos por contradicción que $n_2 = 5$. Si $n_2 = 1$, entonces, ese 2-subgrupo de Sylow de $G$, al que llamaremos $H$, sería el único subgrupo de $G$ de su grado, por lo que sería normal.

Utilizamos ahora que $\mathbb{Q}(\sqrt[5]{5})|\mathbb{Q}$ es una (de hecho, según hemos supuesto, la única) subextensión de grado $5$ de $\mathbb{Q}_f | \mathbb{Q}$. Esto es fácil de comprobar, pues el polinomio
$$g_3(\mathtt{t}) := \mathtt{t}^5 - 5$$
tiene a $\sqrt[5]{5}$ por única raíz real, es mónico, y es irreducible (por el Criterio de Eisenstein con el primo $5$, aplicable a $\mathbb{Q}$ por el Lema de Gauss), luego es el polinomio mínimo sobre $\mathbb{Q}$ de $\sqrt[5]{5}$, y $[\mathbb{Q}(\sqrt[5]{5}) : \mathbb{Q}] = 5$. Así que necesariamente $H = G(\mathbb{Q}_f : \mathbb{Q}(\sqrt[5]{5}))$, por ser $H$ el único subgrupo de orden $4$ de $G$, y $\mathbb{Q}(\sqrt[5]{5}) | \mathbb{Q}$ la única subextensión de grado 5 de $\mathbb{Q}_f | \mathbb{Q}$.

Por la segunda parte del Teorema Fundamental de la Teoría de Galois, el hecho de que $H$ sea normal es equivalente a que la subextensión $\mathbb{Q}(\sqrt[5]{5})|\mathbb{Q}$ sea de Galois. Pero esto es imposible, pues de ser el caso, se tendría que $\mathbb{Q}_{g_3}$, el cuerpo de descomposición del polinomio $g_3$ sobre $\mathbb{Q}$, coincide con $\mathbb{Q}(\sqrt[5]{5})$, pero $\mathbb{Q}(\sqrt[5]{5}) \subset \mathbb{R}$, y sabemos que$g_3$ tiene raíces complejas, luego $\mathbb{Q}_{g_3}$ tiene elementos en $\mathbb{C} \backslash \mathbb{R}$ y no pueden coincidir.

Por tanto, $n_2 = 5$, de modo que $\mathbb{Q}_f | \mathbb{Q}$ tiene $5$ subextensiones de grado $5$, que de hecho son $\mathbb{Q}(\sqrt[5]{5}\zeta^k) | \mathbb{Q}$ con $k=0,1,...,4$.

(v) Lo probamos por contradicción. Si $\sqrt{3} \in \mathbb{Q}(\zeta)$, entonces, como $[\mathbb{Q}(\sqrt{3}):\mathbb{Q}] = 2$, tenemos que tanto $\mathbb{Q}(\sqrt{3})|\mathbb{Q}$ como $\mathbb{Q}(\sqrt{5})|\mathbb{Q}$ son subextensiones de grado $2$ de $\mathbb{Q}(\zeta) | \mathbb{Q}$, y como sabemos que $\sqrt{3} \notin \mathbb{Q}(\sqrt{5})$, necesariamente
$$[\mathbb{Q}(\sqrt{3}, \sqrt{5}):\mathbb{Q}] = [\mathbb{Q}(\sqrt{3}, \sqrt{5}):\mathbb{Q}(\sqrt{5})] \cdot [\mathbb{Q}(\sqrt{5}):\mathbb{Q}] = 2 \cdot 2 = 4$$
Esto implica que $\mathbb{Q}(\sqrt{3}, \sqrt{5}) = \mathbb{Q}(\zeta)$ (pues tenemos $\subset$, y la igualdad se sigue de que ambas extensiones tienen el mismo grado sobre $\mathbb{Q}$), lo cual es imposible porque $\mathbb{Q}(\sqrt{3}, \sqrt{5}) \subset \mathbb{R}$ y $\zeta \in \mathbb{C} \backslash \mathbb{R}$, luego $\sqrt{3} \notin \mathbb{Q}(\zeta)$.

(vi) Sabemos que $[\mathbb{Q}(\zeta):\mathbb{Q}] = \varphi(5) = 4$, y también que $[\mathbb{Q}(\sqrt{3}):\mathbb{Q}] = 2$. Utilizamos el apartado anterior, que nos dice que $\sqrt{3} \notin \mathbb{Q}(\zeta)$, para deducir que $[\mathbb{Q}(\sqrt{3}, \zeta):\mathbb{Q}(\zeta)] = 2$, pues ha de ser mayor que $1$ y menor o igual que $2$ (ya que $[\mathbb{Q}(\sqrt{3}):\mathbb{Q}] = 2$). Gracias a la transitividad del grado, obtenemos que
$$[\mathbb{Q}(\sqrt{3}, \zeta):\mathbb{Q}] = [\mathbb{Q}(\sqrt{3}, \zeta):\mathbb{Q}(\zeta)]\cdot[\mathbb{Q}(\zeta):\mathbb{Q}] = 2 \cdot 4 = 8$$
Esto nos sirve para calcular el grado de la extensión $\mathbb{Q}(\sqrt{3}, \zeta)|\mathbb{Q}(\sqrt{3})$, puesto que
$$[\mathbb{Q}(\sqrt{3}, \zeta):\mathbb{Q}(\sqrt{3})] = \frac{[\mathbb{Q}(\sqrt{3}, \zeta):\mathbb{Q}]}{[\mathbb{Q}(\sqrt{3}):\mathbb{Q}]} = \frac{8}{2} = 4.$$
Por tanto, puesto que el polinomio ciclotómico asociado al primo 5, $\Phi_5$, tiene grado 4, es mónico y se anula en $\zeta$, ha de ser $P_{\mathbb{Q}(\sqrt{3}),\zeta} = \Phi_5(\mathtt{t})=\mathtt{t}^4+\mathtt{t}^3+\mathtt{t}^2+\mathtt{t}+1$.